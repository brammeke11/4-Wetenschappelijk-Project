\documentclass[a4paper,12pt]{article}

\usepackage{fullpage} %Volledige breedte gebruiken
\usepackage[dutch]{babel} %Nederlands

\usepackage{textcomp}
\usepackage{amsmath}

\usepackage{pdfpages} %PDF kunnen importeren

\usepackage[space]{grffile} %Spatie in directory mogelijk

\usepackage{tocloft} %Puntjes in inhoudsopgave
\renewcommand\cftsecleader{\cftdotfill{\cftdotsep}}

\setlength{\parindent}{0pt} %Geen tab bij begin paragraaf

\usepackage[hidelinks]{hyperref} %Klikbare PDF

\usepackage{color} %Tekstkleur

\begin{document}
	\title{Wetenschappelijk project: landschapskantoor}
	\author{Mathias Govaerts\\ Bram de Leege\\ Guido Marree\\ Eric N\'{e}d\'{e}e\\ Bram Smeets\\ Wim Vanderstraeten}
	\date{\today}	
	\maketitle
	\pagenumbering{gobble}
	
	\newpage
	\tableofcontents
	\newpage
	\pagenumbering{arabic}
	
	\section{Inleiding}
	
	\section{Opdracht}
	De opdracht houdt in de verlichting in een bureauomgeving aan te passen aan de noden van de werknemers om zodoende het werkcomfort te verbeteren. Hierin is het de bedoeling om niks te veranderen aan de bestaande armaturen. Er mogen wel lichtbronnen worden bijgeplaatst en de bestaande mogen worden vervangen. Dit met het doel een zo lichtgevende, maar zo zuinig mogelijke oplossing te bekomen. Een nevenopdracht houdt in dat de veiligheid van de armaturen worden verbeterd en een subsidiedossier op te stellen.
	
	\section{Functionele verlichting}
	De functionele verlichting verbeteren is de hoofdzaak. Hierin hadden we twee oplossingen waaruit we de verlichting met
	
	\subsection{Spots in armatuur}
	Bij deze oplossing worden de huidige spots vervangen door alternatieven met een grotere lichtopbrengst en een beter spreiding. Er kunnen ook spots bijgeplaatst worden, als er rekening wordt gehouden met het maximaal te leveren vermogen van de transformatoren. Voor een maximale lichtopbrengst is het aan te raden de spots per twee te plaatsen onder een bepaalde hoek. Bij het uittesten bleek echter dat de schaduw verkregen door jezelf nogal storend was. Daarom plaatsen we om de … meter een paar spots zodat het licht van deze elkaar zouden overlappen, hierdoor is er een minder storende schaduw.
	
	\subsubsection{Specificaties}
	
	\subsubsection{Plattegrond}
	
	\subsection{LED-panelen}
	Een andere oplossing is het bijplaatsen van LED-panelen met verschillende afmetingen afhankelijk van de plaats. Na uittesten blijkt dat hier geen probleem met de schaduw is en dat de panelen een groot genoeg oppervlak verlichten. Twee lange LED-panelen (120 x 30 cm) kunnen een volledig bureau (… x …) goed verlichten.\\
	
	Het nadeel bij deze oplossing is dat het plafond niet overal hetzelfde is, door ongelijke hoogte, de aanwezigheid van akoestische panelen en zonnekoepels. Op het onderstaande plattegrond is echter een haalbare oplossing uitgetekend.
	\subsubsection{Specificaties}
	%Aantallen nakijken!!
	
	De LED-panelen bestaan uit verschillende afmetingen, waarvan enkel de 120 x 30 cm en 60 x 60 cm worden gebruikt. Deze worden aangekocht met een Kleurtemperatuur van 3000K welke het beste is voor een werkomgeving.
	\subsubsection{Plattegrond}
	%Plattegrond met LED-panelen (met Relux?)
	
	\section{Energie-effici\"{e}ntie}
	Om het verbruik te verminderen en de levensduur van de huidige lampjes te vergroten kan gebruikt gemaakt worden van verschillende modellen LED-verlichting. Hierbij is het type aansluiting van belang, omdat dit niet te wijzigen is in de huidige armatuur.
	
	Om een vergelijking te kunnen maken is er een berekening van de kosten en verbuik gemaakt bij de huidige lampen en daarna bij de vernieuwde verlichting. Omdat Mobilant klant is bij OCTA+ is daarvan de prijs per KiloWattuur genomen, namelijk €0,0503/kWh. Ook wordt er vanuit gegaan dat de lampen 8 uren per dag aanstaan en er wordt rekening gehouden met de levensduur van de lampen.
	
	%Nakijken voor definitieve lampen of verschillende vergelijkingen maken tussen verschillende leveranciers en prijzen.
	\section{Offertes}
	%Offertes misschien eerder dan Energie-efficiëntie zodat we definitieve prijzen kunnen gebruiken voor verschillende berekeningen.
	
	Voor elk type verlichting is er een offerte aangevraagd bij minstens 3 bedrijven. Hiertussen zijn ook bedrijven die alle benodigde types leveren, waarvoor een extra offerte is opgemaakt voor een totaallevering.
	\section{Veiligheid}
	De veiligheid van de bestaande armatuur laat te wensen over. De verschillende metalen stangen liggen te dicht bij elkaar. Wanneer ze elkaar raken kan dit kortsluiting veroorzaken. Dit is op verschillende manieren te voorkomen.
	
	%Verschillende manieren uitgebreid uittypen en prijzen vermelden + toepassingsmethode
	
	\section{Subsidies}
	Het opstellen van het subsidiedossier is een additioneel doel. Niet alle verlichting
\end{document}